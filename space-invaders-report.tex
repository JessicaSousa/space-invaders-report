\documentclass[]{book}
\usepackage{lmodern}
\usepackage{amssymb,amsmath}
\usepackage{ifxetex,ifluatex}
\usepackage{fixltx2e} % provides \textsubscript
\ifnum 0\ifxetex 1\fi\ifluatex 1\fi=0 % if pdftex
  \usepackage[T1]{fontenc}
  \usepackage[utf8]{inputenc}
\else % if luatex or xelatex
  \ifxetex
    \usepackage{mathspec}
  \else
    \usepackage{fontspec}
  \fi
  \defaultfontfeatures{Ligatures=TeX,Scale=MatchLowercase}
\fi
% use upquote if available, for straight quotes in verbatim environments
\IfFileExists{upquote.sty}{\usepackage{upquote}}{}
% use microtype if available
\IfFileExists{microtype.sty}{%
\usepackage[]{microtype}
\UseMicrotypeSet[protrusion]{basicmath} % disable protrusion for tt fonts
}{}
\PassOptionsToPackage{hyphens}{url} % url is loaded by hyperref
\usepackage[unicode=true]{hyperref}
\hypersetup{
            pdftitle={Space Invaders - IA},
            pdfauthor={Ivan de Jesus Pereira Pinto},
            pdfborder={0 0 0},
            breaklinks=true}
\urlstyle{same}  % don't use monospace font for urls
\usepackage{natbib}
\bibliographystyle{apalike}
\usepackage{longtable,booktabs}
% Fix footnotes in tables (requires footnote package)
\IfFileExists{footnote.sty}{\usepackage{footnote}\makesavenoteenv{long table}}{}
\usepackage{graphicx,grffile}
\makeatletter
\def\maxwidth{\ifdim\Gin@nat@width>\linewidth\linewidth\else\Gin@nat@width\fi}
\def\maxheight{\ifdim\Gin@nat@height>\textheight\textheight\else\Gin@nat@height\fi}
\makeatother
% Scale images if necessary, so that they will not overflow the page
% margins by default, and it is still possible to overwrite the defaults
% using explicit options in \includegraphics[width, height, ...]{}
\setkeys{Gin}{width=\maxwidth,height=\maxheight,keepaspectratio}
\IfFileExists{parskip.sty}{%
\usepackage{parskip}
}{% else
\setlength{\parindent}{0pt}
\setlength{\parskip}{6pt plus 2pt minus 1pt}
}
\setlength{\emergencystretch}{3em}  % prevent overfull lines
\providecommand{\tightlist}{%
  \setlength{\itemsep}{0pt}\setlength{\parskip}{0pt}}
\setcounter{secnumdepth}{5}
% Redefines (sub)paragraphs to behave more like sections
\ifx\paragraph\undefined\else
\let\oldparagraph\paragraph
\renewcommand{\paragraph}[1]{\oldparagraph{#1}\mbox{}}
\fi
\ifx\subparagraph\undefined\else
\let\oldsubparagraph\subparagraph
\renewcommand{\subparagraph}[1]{\oldsubparagraph{#1}\mbox{}}
\fi

% set default figure placement to htbp
\makeatletter
\def\fps@figure{htbp}
\makeatother

\usepackage{booktabs}
\usepackage{amsthm}

%encoding
%--------------------------------------
\usepackage[T1]{fontenc}
\usepackage[utf8]{inputenc}
%--------------------------------------

%Portuguese-specific commands
%--------------------------------------
\usepackage[brazil]{babel}
%--------------------------------------
\makeatletter
\def\thm@space@setup{%
  \thm@preskip=8pt plus 2pt minus 4pt
  \thm@postskip=\thm@preskip
}
\makeatother

\title{Space Invaders - IA}
\author{Ivan de Jesus Pereira Pinto}
\date{2020-02-12}

\begin{document}
\maketitle

{
\setcounter{tocdepth}{1}
\tableofcontents
}
\chapter*{Resumo}\label{resumo}
\addcontentsline{toc}{chapter}{Resumo}

\begin{center}\rule{0.5\linewidth}{0.5pt}\end{center}

\begin{center}\includegraphics[width=0.5\linewidth]{content/imgs/game} \end{center}

Neste relatório descrevemos as técnicas utilizadas nos agentes do jogo
\emph{space invaders}, além do processo de treinamento.

\chapter{Introdução}\label{intro}

Esse projeto visa o desenvolvimento de técnicas da área de Inteligência
Artificial para o jogo de Space Invaders. Especificamente almeja-se a
implementação de IAs para as naves inimigas. A maior parte da pesquisa
nessa área se dá no desenvolvimento de Agentes que aprendem a jogar
contra o ambiente, no estilo de um processo de decisão de Markov abaixo:

O SpaceInvaders aqui desenvolvido almeja no entanto politicas de
controle para o ambiente, que seriam os inimigos. O jogador(humano) se
torna o ambiente nesse caso. Para que fosse possível a implementação das
técnicas de IA, foi necessário o desenvolvimento do jogo, de um
simulador, e de um surrogate para o jogador. Os objetivos deste trabalho
são listados a seguir:

\begin{enumerate}
\def\labelenumi{\arabic{enumi}.}
\tightlist
\item
  Construção do jogo de Space Invaders eficiente em C.
\item
  Construção de um Forward Simulator
\item
  Desenvolvimento de um surrugate Player que substitua o humano em tempo
  de planning ou treino.
\item
  Implementação de Técnicas de IA (Planning e Aprendizado)
\end{enumerate}

\chapter{Trabalhos relacionados}\label{trabalhos-relacionados}

Here is a review of existing methods.

\chapter{Metodologia}\label{methods}

We describe our methods in this chapter.

\chapter{Resultados e discussão}\label{resultados-e-discussuxe3o}

Some \emph{significant} applications are demonstrated in this chapter.

\section{Example one}\label{example-one}

\section{Example two}\label{example-two}

\chapter{Conclusões e trabalhos
futuros}\label{conclusuxf5es-e-trabalhos-futuros}

We have finished a nice book.

\bibliography{book.bib,packages.bib}

\end{document}
